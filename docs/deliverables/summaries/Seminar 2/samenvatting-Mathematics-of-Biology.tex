\documentclass{article}

\usepackage{amsmath,fullpage}

\begin{document}
	\section{Rates of chemical reactions}
		The \textbf{equilibrium position} of a reversible reaction, the point at which the rate of formation equals the rate of dissociation, is determined by a combination of the forward and reverse \textbf{rate constants}.\\
		\textbf{For unidirectional reactions ($A + B \rightarrow X + Y$):} If $X$ is the number of molecules of any reactant, the \textbf{rate of disappearance} equals that of any other reactant and is decribed by $\frac{dX}{dt}$. Due to the \emph{Conservation of Mass} principle, this rate must be inversely proportional to the rate of appearance of any and every individual product. To calculate the amount of molecules of a species at any time $t$, the rate of (dis)appearance can be integrated over the interval $[0,t]$ for any reactant or product.\\
		$\frac{dX}{dt} = \frac{dY}{dt} = -\frac{dA}{dt} = -\frac{dB}{dt}$ and $X(t) - X_0 = Y(t) - Y_0 = -A(t) + A_0 = -B(t) - B_0$\\
		To know any one of the amounts is to know all, but to know any one of them, the speed of the reaction must be known. According to the \emph{Law of Mass Action} the \textbf{rate of simultaneous combination} (second derivative of $X$ as a function of $t$) of two chemicals is proportional to the product of their concentration $X]$ and can be described as $k[A][B]$, where $k$ is the \textbf{constant of proportionality}.\\
		Assuming the concentration of a species at any moment is either determined by the fixed volume of the medium (closed reaction vessel) or by the inflowing reactants (open reaction vessel), the concentration may either refer to the relative number of molucules of a species within a solution when the species dissolves in the medium, or the ratio of precipitated molecules (only product) to the medium. This notion of concentration at any moment of species $X$ is $x(t)$.\\
		Combining the equations of amount and concentration, and the \emph{Law of Mass Action}, the differential equation with initial value $x(0) = x_0$ is found, which stationary points are given by setting the right-hand side to zero and solving: $\frac{dx}{dt} = kab = k(a + x_0 - x)(b + x_0 - x)$ and $x = a_0 + x_0$ or $x = b_0 + x_0$. By solving the general equation, the \textbf{progression of the reaction as a function of time} is found.\\
		\textbf{For reversible reactions:} The principles are the same, but there is a forward constant $k_1$ and a backward constant $k_{-1}$ for each chemical. If the forward and backward reactions are independent, the rate of change is the sum of the effects of each reaction. If the forward and reverse constants are distinct, the equation describing the stationary points is quadratic and thus easily factorized, with real roots if $x_0 \leq y_0$. The progression of the reaction can be found in a similar way as that of a unidirectional reaction.
	\section{Enzyme Kinetics}
		The reaction from substrate $S$ in combination with enzyme $E$ to product $P$ is performed by forming an \textbf{enzyme-substrate complex} $C$ which decomposes into product and enzyme. Due to the typically small amount of enzyme compared to subtrate, the conversion rate is limited when the enzyme becomes saturated with substrate (\textbf{enzyme saturation}). No enzyme is destroyed or produced, so $e_0$ is the initial and total amount of enzyme. The combination $k_M = \frac{k_{-1} + k_2}{k_1}$ of rate constants is known as the \textbf{Michaelis-Menten constant} and it determines the concentration of complex $c$ at low substrate concentrations. At high substrate levels, the complex reaches a relatively invariant concentration $c_{Eff}$, and because $c = \frac{se_0}{k_M + s}$ becomes dependent on $s$, $c_{Eff} \approx e0$.\\
		The velocity $v$ of the reaction is the appearance rate of the product and so at high substrate levels $v_{max} = k_2e_0$. The initial velocity is found by substituting $v_{max}$ into the \emph{Law of Mass action} for $t = 0$. By expirimentally measuring the reaction rate for various substrate concentrations a sketch of the graph can be made and working from the graph $k_M$ can be determined, being the substrate concentration where the reaction rate is half maximal.
\end{document}
