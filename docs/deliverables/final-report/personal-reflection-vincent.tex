Despite the fact Biology has always been one of my fields of interest I didn't do a lot with it up until this project. The context seminars definitely spiked my interest, especially because it brought biology and informatics together in the form of Bioinformatics. I never thought much about the application of informatics in other fields of study, so this was almost entirely new to me. I really enjoyed the modelling of a cell by means of programming, this shows the opportunities and capabilities of Bioinformatics, definitely something to remember and take into account when choosing a master.

Before the project started I had limited to no experience\footnote{Limited working experience with Git, no experience with Ruby, Rails, Coffeescript (for Javascript)and SASS (for CSS), no working experience with model-view-controller (MVC) architecture, no working experience with SCRUM} with almost all of technologies used during the project and knew it was going to be an interesting run. This had some drawbacks as well as it's advantages. A drawback was, especially at the beginning of the project, a lot of tasks took a lot more time than actually necessary. This was because I had to invest a lot of time learning these new technologies. At the same time there were also courses to keep up with, so this increased the workload. Nevertheless, learning all these new things has been a wonderful opportunity and I have by no means any regret of choosing this project and the way in which we worked on it.

During the project we made use of Planbox for agile project management, it's a great way to  manage the product and it's yet to be implemented features. However, it took me a while to figure out how to use it properly. For instance, at the beginning of the project when I was working on a task I would forget to start the timer and when finishing the task would just roughly guess the amount of time it took me. The same problem I had with using git, when working on some tasks I would just do my work and sometimes not even commit, this led to some problems when others were also working in the same branch. It took me a while to get used to the work-flow of recording the time when working on a task, committing after completing a task and then discuss changes made. Now I have a much better grasp on how to work with these kind of tools and use them to my advantage when working on other projects in the future.

Without a lot of experience and trying to make up for that by showing what I could do just by myself, I sometimes got in my own way. I would get stuck somewhere and would be frustrated if things weren't working out. However, there was always some group member to help me out in the end, this showed me the importance of communicating problems early and not to make a big deal about having to ask someone for help.

Overall, this has been an amazing project with a lot of invaluable lessons, not only from the assistants/teachers but also from the other group members.\\
