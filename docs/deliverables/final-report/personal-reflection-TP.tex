\documentclass{article}

\usepackage{fullpage}

\title{TI2800 - Contextproject - Programming Life - Personal Reflection}
\author{T. Phan}
\date{\today}

\begin{document}
	\maketitle
	
	Before the project began, I had no idea what would be in store for me. I didn't know what bioinformatics was and, to be fair, it wasn't my first choice either. But I have zero regrets, as I've rediscovered my love for informatics and learned to view it from a different perspective. Thanks to the project, bioinformatics is also something I'll take into account when I'm choosing a master.\\

	The project taught me my limitations of an undergraduate, as my skills were more on the lacking side that I hoped. And that's okay. I learned to take stuff at my own pace, instead of constantly comparing my capabilities to others. That also reassures me that computer science is still the future for me, as for the past year, I had serious doubts about my choice and skills. I considered it a race, and I was losing it. I was constantly being amazed by other programmers my age, how they were completely out of my league and doubted my future. Due to help from a few team members, I finally realised, even though it sounds obvious, that I'm going to college for my own good. Not for my family, friends or future boss, but for myself. They reached this level of skill by dedication and hard work and not by some funny whim of nature, it's just that some people are quicker on the uptake than others and I'm fine with it.\\

	After experiencing what bioinformatics can do, I've been curious about the application of informatics in the world, combining different fields and forming a whole new field.\\

	As we worked with a whole new language in this project, I've learned to work outside my comfort zone, getting used to the new tools and developing environment. It has also urged me to learn more about said language, Ruby (on Rails), and this project made me want to work on personal projects in this language too.\\

	Lastly, this project has been a confirmation for what I want to do later in life as a computer scientist. I definitely belong in the higher level group of computer scientists: focussing more on developing software than, for example, optimising the programming languages. One of the things I want do to is to be able to develop software to help other fields effectively accomplish their goals.

\end{document}
