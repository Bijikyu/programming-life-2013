Let me start off by being honest and admitting that my opinion of this particular context and the assignment was not especially high. I always like to build something that will or could actually be used. The game context would have provided me with a project to create something like that, and for that I listed it as my first choice. The Programming Life context didn’t seem to offer such a chance. However, I still had my reasons for listing this context as my second choice. I find the scientific field of microbiology a very interesting one, even though I never took biology in high school beyond the 3rd grade. \\

Also, before being enrolled into a particular context, I had the desire to step away from the web for a change. I have built websites for a living and I wanted a ‘change of scenery’, and I believe that was the case for more members of the team. Interestingly enough, in the first meetings the team held we decided that we were going to build a web app. To keep the assignment a challenge, we decided we decided we would get our kicks by using platforms, tools and languages we hadn’t used before. And that was a blessing. \\

The best discovery for me is the language CoffeeScript. It’s an object-oriented, Ruby inspired language that is converted - by the server - into JavaScript which can be run in a normal web browser. It was able to turn my dislike of developing for the web around and made me find again the fun that I have when I really enjoy what I’m doing. Throughout the project I found that the more I understood CoffeeScript, the more fun I had and the more time and effort I put into the project. \\

Besides the new tools, this was the first time I - and I believe any of our team - have actually employed the Scrum workflow. It’s a very nice way to do your project planning, although we didn’t put all the effort into it to follow the process it by the letter. We didn’t assign a designated Scrum master but instead rotated the roles so all team members would gain some experience into this process. Also, we didn’t maintain the sprint log an backlog as well as we could have. Estimations for sprint log items also gained a lot of accuracy through the sprints. \\

I realize during the project I have grown as a software developer and as a team member. The team was awesome: we were all eager to learn new things, not afraid to ask questions and clever enough to develop out-of-the-box solutions. I’ve also learnt a lot about the context: the provided seminars provided me with a lot of food for thought and have inspired many hours of browsing and clicking through Wikipedia in the hunt for more knowledge. Nature works in incredible ways, and the basic knowledge provided in the seminars have enabled me to further grasp some of the intriguing details.
